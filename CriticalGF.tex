\documentclass[]{article}
\usepackage{lmodern}
\usepackage{amssymb,amsmath}
\usepackage{ifxetex,ifluatex}
\usepackage{fixltx2e} % provides \textsubscript
\ifnum 0\ifxetex 1\fi\ifluatex 1\fi=0 % if pdftex
  \usepackage[T1]{fontenc}
  \usepackage[utf8]{inputenc}
\else % if luatex or xelatex
  \ifxetex
    \usepackage{mathspec}
  \else
    \usepackage{fontspec}
  \fi
  \defaultfontfeatures{Ligatures=TeX,Scale=MatchLowercase}
\fi
% use upquote if available, for straight quotes in verbatim environments
\IfFileExists{upquote.sty}{\usepackage{upquote}}{}
% use microtype if available
\IfFileExists{microtype.sty}{%
\usepackage[]{microtype}
\UseMicrotypeSet[protrusion]{basicmath} % disable protrusion for tt fonts
}{}
\PassOptionsToPackage{hyphens}{url} % url is loaded by hyperref
\usepackage[unicode=true]{hyperref}
\hypersetup{
            pdfborder={0 0 0},
            breaklinks=true}
\urlstyle{same}  % don't use monospace font for urls
\usepackage{longtable,booktabs}
% Fix footnotes in tables (requires footnote package)
\IfFileExists{footnote.sty}{\usepackage{footnote}\makesavenoteenv{long table}}{}
\IfFileExists{parskip.sty}{%
\usepackage{parskip}
}{% else
\setlength{\parindent}{0pt}
\setlength{\parskip}{6pt plus 2pt minus 1pt}
}
\setlength{\emergencystretch}{3em}  % prevent overfull lines
\providecommand{\tightlist}{%
  \setlength{\itemsep}{0pt}\setlength{\parskip}{0pt}}
\setcounter{secnumdepth}{0}
% Redefines (sub)paragraphs to behave more like sections
\ifx\paragraph\undefined\else
\let\oldparagraph\paragraph
\renewcommand{\paragraph}[1]{\oldparagraph{#1}\mbox{}}
\fi
\ifx\subparagraph\undefined\else
\let\oldsubparagraph\subparagraph
\renewcommand{\subparagraph}[1]{\oldsubparagraph{#1}\mbox{}}
\fi

% set default figure placement to htbp
\makeatletter
\def\fps@figure{htbp}
\makeatother

\usepackage{setspace}
\doublespacing
\usepackage[vmargin=1in,hmargin=1in]{geometry}
\usepackage{graphicx}
\usepackage{subfig}
\usepackage{graphicx}
\usepackage{tabularx}
\usepackage{float}
\usepackage{siunitx}
\usepackage[running]{lineno}
\linenumbers

\date{}

\begin{document}

\section{Power laws and critical fragmentation in global
forests}\label{power-laws-and-critical-fragmentation-in-global-forests}

\textbf{Leonardo A. Saravia} \textsuperscript{1} \textsuperscript{3},
\textbf{Santiago R. Doyle} \textsuperscript{1}, \textbf{Ben
Bond-Lamberty}\textsuperscript{2}

\begin{enumerate}
\def\labelenumi{\arabic{enumi}.}
\item
  Instituto de Ciencias, Universidad Nacional de General Sarmiento, J.M.
  Gutierrez 1159 (1613), Los Polvorines, Buenos Aires, Argentina.
\item
  Pacific Northwest National Laboratory, Joint Global Change Research
  Institute at the University of Maryland--College Park, 5825 University
  Research Court \#3500, College Park, MD 20740, USA
\item
  Corresponding author e-mail:
  \href{mailto:lsaravia@campus.ungs.edu.ar}{\nolinkurl{lsaravia@campus.ungs.edu.ar}}
\end{enumerate}

\textbf{keywords}: Forest fragmentation, early warning signals,
percolation, power-laws, MODIS, critical transitions

\textbf{Running title}: Critical fragmentation in global forest

\textbf{Author contributions}: LAS designed the study, SRD LAS performed
modelling work, BBL review methods, and LAS SRD BBL wrote the manuscript

\textbf{Data availability}: all the raw data is from public
repositories, the data supporting the results is archived at figshare
public repository and the data DOI is included at in the article.

\textbf{Competing interests}: The author(s) declare no competing
interests.

\newpage

\subsection{Abstract}\label{abstract}

The replacement of forest areas with human-dominated landscapes usually
leads to fragmentation, altering the structure and function of the
forest. Here we studied the dynamics of forest patch sizes at a global
level, examining signals of a critical transition from an unfragmented
to a fragmented state, using the MODIS vegetation continuous field. We
defined wide regions of connected forest across continents and big
islands, and combined five criteria, including the distribution of patch
sizes and the fluctuations of the largest patch over the last sixteen
years, to evaluate the closeness of each region to a fragmentation
threshold. Regions with the highest deforestation rates---South America,
Southeast Asia, Africa---all met these criteria and may thus be near a
critical fragmentation threshold. This implies that if current forest
loss rates are maintained, wide continental areas could suddenly
fragment, triggering extensive species loss and degradation of
ecosystems services.

\newpage

\subsection{Introduction}\label{introduction}

Forests are among the most important biomes on earth, providing habitat
for a large proportion of species and contributing extensively to global
biodiversity\textsuperscript{1}. In the previous century, human
activities have influenced global bio-geochemical
cycles\textsuperscript{2,3}, with one of the most dramatic changes being
the replacement of 40\% of Earth's formerly biodiverse land areas with
landscapes that contain only a few species of crop plants, domestic
animals and humans\textsuperscript{4}. These local changes have
accumulated over time and now constitute a global
forcing\textsuperscript{5}. Another global scale forcing that is tied to
habitat destruction is fragmentation, which is defined as the division
of a continuous habitat into separated portions that are smaller and
more isolated. Fragmentation produces multiple interwoven effects:
reductions of biodiversity between 13\% and 75\%, decreasing forest
biomass, and changes in nutrient cycling\textsuperscript{6}. The effects
of fragmentation are not only important from an ecological point of view
but also that of human activities, as ecosystem services are deeply
influenced by the level of landscape
fragmentation\textsuperscript{7--9}.

Ecosystems harbour hundreds of populations interacting through complex
networks and present feedbacks at different levels of
organization\textsuperscript{10,11}, external forcings can produce
abrupt changes from one state to another, called critical
transitions\textsuperscript{12}. Complex systems can experience two
general classes of critical transitions\textsuperscript{13}. In
so-called first-order transitions, a catastrophic regime shift that is
mostly irreversible occurs because of the existence of alternative
stable states\textsuperscript{14}. This class of transitions is
suspected to be present in a variety of ecosystems such as lakes,
woodlands, coral reefs\textsuperscript{14}, semi-arid
grasslands\textsuperscript{15}, and fish
populations\textsuperscript{16}. They can be the result of positive
feedback mechanisms\textsuperscript{17}; for example, fires in some
forest ecosystems were more likely to occur in previously burned areas
than in unburned places\textsuperscript{18}.

The other class of critical transitions are second order
transitions\textsuperscript{19}. In these cases, there is a narrow
region where the system suddenly changes from one domain to another in a
continuous and reversible way. Such transitions have been suggested for
tropical forests\textsuperscript{20,21}, semi-arid mountain
ecosystems\textsuperscript{22}, and tundra
shrublands\textsuperscript{23}. The transition happens at a critical
point where we can observe scale-invariant fractal structures
characterized by power law patch distributions\textsuperscript{24}.

The spatial phenomena observed in continuous critical transitions are
related to connectivity, a fundamental property of general systems and
ecosystems from forests\textsuperscript{25} to marine
ecosystems\textsuperscript{26} and the whole
biosphere\textsuperscript{27}. When a system goes from a fragmented to a
connected state we say that it percolates\textsuperscript{13}.
Percolation implies that there is a path of connections that involve the
whole system. Thus we can characterize two domains or phases: one
dominated by short-range interactions which does not allow information
(used in a broad sense, e.g.~species dispersal or movement) to spread
through the whole system, and another in which long-range interactions
are possible and information can spread throughout the system. Thus,
there is a critical ``percolation threshold'' between the two phases,
and the system could be driven close to or beyond this point by an
external force (called tunning parameter); climate change, deforestation
and forest fire are the main forces that could be the drivers of such a
phase change in contemporary forests\textsuperscript{3,6}. There are
several applications of this concept in ecology: species' dispersal
strategies are influenced by percolation thresholds in three-dimensional
forest structure\textsuperscript{28}, and it has been shown that species
distributions also have percolation thresholds\textsuperscript{29}. This
implies that pushing the system below the percolation threshold could
produce a biodiversity collapse\textsuperscript{30--32}; conversely,
being in a connected state (above the threshold) could accelerate the
invasion of the forest into prairie\textsuperscript{23,33}.

One of the main challenges with systems that can experience critical
transitions---of any kind---is that the value of the critical threshold
is not known in advance. In addition, because near the critical point a
small change can precipitate a state shift in the system, they are
difficult to predict. Several methods have been developed to detect if a
system is close to the critical point, e.g.~a deceleration in recovery
from perturbations, or an increase in variance in the spatial or
temporal pattern\textsuperscript{34--37}.

The existence of a critical transition between two states has been
established for forest at a global scale in different
works\textsuperscript{38--40}. It is generally believed that this
constitutes a first order catastrophic transition. There are several
processes that can convert a catastrophic transition to a second order
transition\textsuperscript{17}. These include stochasticity, such as
demographic fluctuations, spatial heterogeneities, and/or dispersal
limitation. All these components are present in forest around the
globe\textsuperscript{41--43}, and thus continuous transitions might be
more probable than catastrophic transitions. Moreover, there is evidence
of recovery in systems that supposedly suffered an irreversible
transition produced by overgrazing\textsuperscript{44,45} and
desertification\textsuperscript{46}. From this basis, we applied indices
derived from second-order transitions to global forest cover dynamics.

In this study, our objective is to look for evidence that forests around
the globe are near continuous critical points that represent a
fragmentation threshold. We use the framework of percolation to first
evaluate if forest patch distribution at a continental scale is
described by a power law distribution and then examine the fluctuations
of the largest patch. The advantage of using data at a continental scale
is that for very large systems the transitions are very
sharp\textsuperscript{13} and thus much easier to detect than at smaller
scales, where noise can mask the signals of the transition.

\subsection{Methods}\label{methods}

\subsubsection{Study areas definition}\label{study-areas-definition}

We analysed mainland forests at a continental scale, covering the whole
globe, by delimiting land areas with a near-contiguous forest cover,
separated from each other by large non-forested areas. We defined three
forest regions in America: South America temperate forest (SAT),
subtropical and tropical forest up to Mexico (SAST), USA and Canada
forest (NA). Europe and North Asia are one region (EUAS), then
South-east Asia (SEAS), Africa (AF), and Australia (OC). We also
analysed islands larger than \si{10^5 km^2}. This criterion to delimit
regions is based on percolation theory that assumes some kind of
connectivity in the study area (Appendix Table S2, figure S1-S6).

\subsubsection{Forest patch
distribution}\label{forest-patch-distribution}

We studied forest patch distribution in each area from 2000 to 2015
using the MODerate-resolution Imaging Spectroradiometer (MODIS)
Vegetation Continuous Fields (VCF) Tree Cover dataset version
051\textsuperscript{47}. This dataset is produced at a global level with
a 231-m resolution on an annual basis. There are several definitions of
forest based on percent tree cover\textsuperscript{48} and these are
used to convert the percentage tree cover to a binary image of forest
and non-forest pixels; we chose a range from 20\% to 40\% threshold in
5\% increments. This range is centred in the definition used by the
United Nations' International Geosphere-Biosphere
Programme\textsuperscript{49}, studies of global
fragmentation\textsuperscript{6} and includes the range used in other
studies of critical transitions\textsuperscript{50}. Using this range we
avoid the errors produced by low discrimination of MODIS VCF between
forest and dense herbaceous vegetation at low forest cover and the
saturation of MODIS VCF in dense forests\textsuperscript{51}. We
repeated all analyses across this set of thresholds, except in some
specific cases described below. Patches of contiguous forest were
determined in the binary image by grouping connected forest pixels using
a neighbourhood of 8 forest units (Moore neighbourhood). The MODIS VCF
product does not discriminate between tree types, and so besides natural
forest it includes plantations of tree crops like rubber, oil palm,
eucalyptus and other managed stands\textsuperscript{52}. Even though
datasets with lower resolutions than MODIS VCF, like MODIS Land Cover
Type, have been used to study fragmentation\textsuperscript{53},
products with higher resolution that describe forest cover also
exist\textsuperscript{54}. As we analyse the time series of forest
patches, we cannot use the 54 dataset which has a very limited temporal
resolution (years 2000 \& 2012).

\subsubsection{Percolation theory}\label{percolation-theory}

A more in-depth introduction to percolation theory can be found
elsewhere\textsuperscript{24} and a review from an ecological point of
view is available\textsuperscript{55}. Here, to explain the basic
elements of percolation theory we formulate a simple model: we represent
our area of interest by a square lattice and each site of the lattice
can be occupied---e.g.~by forest---with a probability \(p\), thus the
lattice will be more occupied when \(p\) is greater, but the sites are
randomly distributed. We defined patches with the same 8 sites
neighbourhood previously mentioned. When \(p\) is increased from low
values, a patch that connects the whole lattice suddenly appears. At
this point, it is said that the system percolates and the value of \(p\)
is the critical point \(p_c\).

Thus percolation is characterized by two well-defined phases: the
unconnected phase when \(p<p_c\), in which species cannot travel far
inside the forest, as it is fragmented; in a general sense, information
cannot spread. The second is the connected phase when \(p>p_c\), species
can move inside a forest patch from side to side of the lattice,
i.e.~information can spread over the whole area. Near the critical
point, several scaling laws arise: the structure of the patch that spans
the area is fractal, the size distribution of the patches is power-law,
and other quantities also follow power-law scaling\textsuperscript{24}.

The value of the critical point \(p_c\) depends on the geometry of the
lattice and on the definition of the neighbourhood, but other power-law
exponents only depend on the lattice dimension. Close to the critical
point, the distribution of patch sizes is:

\begin{enumerate}
\def\labelenumi{(\arabic{enumi})}
\tightlist
\item
  \(n_s(p_c) \propto s^{-\alpha}\)
\end{enumerate}

where \(n_s(p)\) is the number of patches of size \(s\). The exponent
\(\alpha\) does not depend on the details of the model and it is called
universal\textsuperscript{24}. These scaling laws can be applied to
landscape structures that are approximately random, or correlated over
short distances\textsuperscript{56}. In physics, this is called
``isotropic percolation universality class'', and corresponds to an
exponent \(\alpha=2.05495\). If we observe that the patch size
distribution has another exponent it will not belong to this
universality class and other mechanisms should be invoked to explain it.
Percolation can also be generated by models that have some kind of
memory\textsuperscript{57,58}: for example, a patch that has been
exploited for many years will recover differently than a recently
deforested forest patch. In this case, the system could belong to a
different universality class, or in some cases, there is no
universality, in which case the value of \(\alpha\) will depend on the
parameters and details of the model\textsuperscript{59}.

To illustrate these concepts, we conducted simulations with a simple
forest model with only two states: forest and non-forest. This type of
model is called a ``contact process'' and was introduced for
epidemics\textsuperscript{60} but has many applications in
ecology\textsuperscript{19,56}(see supplementary data, gif animations).

\subsubsection{Patch size distributions}\label{patch-size-distributions}

We fitted the empirical distribution of forest patches calculated for
each of the percentage forest cover thresholds we defined. We fit four
distributions using maximum likelihood\textsuperscript{61,62}:
power-law, power-law with exponential cut-off, log-normal, and
exponential. We assumed that the patch size distribution is a continuous
variable discretised by the remote sensing data acquisition procedure.

The power-law distribution requires a lower bound for its scaling
behaviour that is estimated from the data by maximizing the
Kolmogorov-Smirnov (KS) statistic between the empirical and fitted
cumulative distribution functions\textsuperscript{62}. For the
log-normal model, we constrain the \(\mu\) parameter to positive values,
this parameter controls the mode of the distribution and when is
negative most of the probability density of the distribution lies
outside the range of the forest patch size data\textsuperscript{63}.

To select the best model we calculated the corrected Akaike Information
Criteria (\(AIC_c\)) and Akaike weights for each
model\textsuperscript{64}. Akaike weights (\(w_i\)) are the weight of
evidence in favour of model \(i\) being the best model among the
candidate set of \(N\) models. Additionally, we computed a likelihood
ratio test\textsuperscript{62,65} of the power law model against the
other distributions. We calculated bootstrapped 95\% confidence
intervals\textsuperscript{66} for the parameters of the best model,
using the bias-corrected and accelerated (BCa)
bootstrap\textsuperscript{67} with 10000 replications.

\subsubsection{Largest patch dynamics}\label{largest-patch-dynamics}

The largest patch connects the highest number of sites in the area and
has been used to indicate fragmentation\textsuperscript{25,68}. The size
of the largest patch \(S_{max}\) has been studied in relation to
percolation phenomena\textsuperscript{24,69,70} but seldom used in
ecological studies (but see\textsuperscript{56}). When the system is in
a connected state (\(p>p_c\)) the landscape is almost insensitive to the
loss of a small fraction of forest, but close to the critical point a
minor loss can have important effects\textsuperscript{19,55}, because at
this point the largest patch will have a filamentary structure,
i.e.~extended forest areas will be connected by thin threads. Small
losses can thus produce large fluctuations.

To evaluate the fragmentation of the forest the proportion of the
largest patch against the total area can be
calculated\textsuperscript{71}. The total area of the regions we are
considering (Appendix S4, figures S1-S6) may not be the same as the
total area that the forest could potentially occupy, and thus a more
accurate way to evaluate the weight of \(S_{max}\) is to use the total
forest area, which can be easily calculated by summing all the forest
pixels. We calculate the proportion of the largest patch for each year,
dividing \(S_{max}\) by the total forest area of the same year:
\(RS_{max} = S_{max}/\sum_{i}S_i\). When the proportion \(RS_{max}\) is
large (more than 60\%) the largest patch contains most of the forest so
there are fewer small forest patches and the system is probably in a
connected phase. Conversely, when it is low (less than 20\%), the system
is probably in a fragmented phase\textsuperscript{72}. To define if a
region will be in a connected or unconnected state we used the
\(RS_{max}\) of the highest (i.e., most conservative) threshold of 40\%,
that represents the most dense area of forest within our chosen range.
We assume that there are two alternative states for the critical
transition---the forest could be fragmented or unfragmented. If
\(RS_{max}\) is a good indicator forest's state, its distribution of
frequencies should be bimodal\textsuperscript{15}, so we apply the
Hartigan's dip test that measures departures from
unimodality\textsuperscript{73}.

To evaluate if the forest is near a critical transition, we calculate
the fluctuations of the largest patch
\(\Delta S_{max}=S_{max}(t)-\langle S_{max} \rangle\), using the same
formula for \(RS_{max}\). To characterize fluctuations we fitted three
empirical distributions: power-law, log-normal, and exponential, using
the same methods described previously. We expect that large fluctuations
near a critical point have heavy tails (log-normal or power-law) and
that fluctuations far from a critical point have exponential tails,
corresponding to Gaussian processes\textsuperscript{74}. We also apply
the likelihood ratio test explained previously\textsuperscript{62,65};
if the p-values obtained to compare the best distribution against the
others are not significant we can not decide which is the best model. We
generated animated maps showing the fluctuations of the two largest
patches at 30\% threshold, to aid in the interpretations of the results.

A robust way to detect if the system is near a critical transition is to
analyse the increase in variance of the density\textsuperscript{75}---in
our case `density' is the total forest cover divided by the area. But
the variance increase in density appears when the system is very close
to the transition\textsuperscript{59}, and thus practically it does not
constitute an early warning indicator. An alternative is to analyse the
variance of the fluctuations of the largest patch \(\Delta S_{max}\):
the maximum is attained at the critical point but a significant increase
occurs well before the system reaches the critical
point\textsuperscript{59,72}. In addition, before the critical
fragmentation, the skewness of the distribution of \(\Delta S_{max}\) is
negative, implying that fluctuations below the average are more
frequent. We characterized the increase in the variance using quantile
regression: if variance is increasing the slopes of upper or/and lower
quartiles should be positive or negative.

Statistical analyses were performed using the GNU R version
3.3.0\textsuperscript{76}, to fit patch size distributions we used the
Python package powerlaw\textsuperscript{77}. For quantile regressions,
we used the R package quantreg\textsuperscript{78}, and for image
processing we used MATLAB r2015b. The ``bwconncomp'' MATLAB function,
which implements a flood-fill algorithm, was used to identify individual
patches from binary images. The complete source code for image
processing, statistical analysis and patch size data are available at
figshare \url{http://dx.doi.org/10.6084/m9.figshare.4263905}.

\subsection{Results}\label{results}

Figure 1 shows an example of the distribution of the biggest 200 patches
for years 2000 and 2014. This distribution is highly variable; the
biggest patch usually maintains its spatial location, but sometimes it
breaks and then large temporal fluctuations in its size are observed, as
we will analyse below. Smaller patches can merge or break more easily so
they enter or leave the list of 200, and this is why there is a colour
change across years.

The power law distribution was selected as the best model in 99\% of the
cases (Figure S7). In a small number of cases (1\%), the power law with
exponential cutoff was selected, but the value of the parameter
\(\alpha\) was similar by \(\pm 0.03\) to the pure power law (Table S1
and model fit data table), and thus we used the power law parameters for
these cases (region EUAS3, SAST2) as well. In finite-size systems, the
favoured model should be the power law with exponential cut-off because
the power-law tails are truncated to the size of the
system\textsuperscript{24}. We observe that when the pure power-law
model is the best model, the 64\% of likelihood ratio tests against
power law with exponential cutoff are not significant
(p-value\textgreater{}0.05). Instead, the likelihood ratio test clearly
differentiates the power law model from the exponential model (100\%
cases p-value\textless{}0.05), and the log-normal model (90\% cases
p-value\textless{}0.05).

The global mean of the power-law exponent \(\alpha\) is 1.967 and the
bootstrapped 95\% confidence interval is 1.964 - 1.970. The global
values for each threshold are different, because their confidence
intervals do not overlap, and their range goes from 1.90 to 2.01 (Table
S1). Analysing the biggest regions (Figure 1, Table S2) the northern
hemisphere regions (EUAS1 \& NA1) have similar values of \(\alpha\)
(1.97, 1.98), pantropical areas have different \(\alpha\) with greatest
values for South America (SAST1, 2.01) and in descending order Africa
(AF1, 1.946) and Southeast Asia (SEAS1, 1.895). With greater \(\alpha\)
the fluctuations of patch sizes are lower and vice
versa\textsuperscript{79}.

We calculated the total forest areas and the largest patch \(S_{max}\)
by year for different thresholds, and as expected these two values
increase for smaller thresholds (Table S3). We expect fewer variations
in the largest patch relative to total forest area \(RS_{max}\) (Figure
S9); in ten cases it stayed near or higher than 60\% (EUAS2, NA5, OC2,
OC3, OC4, OC5, OC6, OC8, SAST1, SAT1) over the 25-35 range or more. In
four cases it stayed around 40\% or less, at least over the 25-30\%
range (AF1, EUAS3, OC1, SAST2), and in six cases there is a crossover
from more than 60\% to around 40\% or less (AF2, EUAS1, NA1, OC7, SEAS1,
SEAS2). This confirms the criteria of using the most conservative
threshold value of 40\% to interpret \(RS_{max}\) with regard to the
fragmentation state of the forest. The frequency of \(RS_{max}\) showed
bimodality (Figure S10) and the dip test rejected unimodality (D =
0.0416, p-value = 0.0003), which also implies that \(RS_{max}\) is a
good index to study the fragmentation state of the forest.

The \(RS_{max}\) for regions with more than \(10^{7}\) \si{km^2} of
forest is shown in figure 3. South America tropical and subtropical
(SAST1) is the only region with an average close to 60\%, the other
regions are below 30\%. Eurasia mainland (EUAS1) has the lowest value
near 20\%. For regions with less total forest area (Figure S10, Table
S3), Great Britain (EUAS3) has the lowest proportion less than 5\%, Java
(OC7) and Cuba (SAST2) are under 25\%, while other regions such as New
Guinea (OC2), Malaysia/Kalimantan (OC3), Sumatra (OC4), Sulawesi (OC5)
and South New Zealand (OC6) have a very high proportion (75\% or more).
Philippines (SEAS2) seems to be a very interesting case because it seems
to be under 30\% until the year 2007, fluctuates around 30\% in years
2008-2010, then jumps near 60\% in 2011-2013 and then falls again to
30\%, this seems an example of a transition from a fragmented state to
an unfragmented one (figure S11).

The Likelihood ratio test was not significant for the distributions of
largest patch fluctuations \(\Delta RS_{max}\) and \(\Delta S_{max}\).
Thus we cannot determine with confidence which is the best distribution.
In only one case was the distribution selected by the Akaike criteria
confirmed as the correct model for relative and absolute fluctuations
(Table S4).

The animations of the two largest patches (see supplementary data,
largest patch gif animations) qualitatively shows the nature of
fluctuations and whether the state of the forest is connected or not. If
the largest patch is always the same patch over time, the forest is
probably not fragmented; this happens for regions with \(RS_{max}\) of
more than 40\% such as AF2 (Madagascar), EUAS2 (Japan), NA5
(Newfoundland) and OC3 (Malaysia). In regions with \(RS_{max}\) between
40\% and 30\%, the identity of the largest patch could change or stay
the same in time. For OC7 (Java) the largest patch changes and for AF1
(Africa mainland) it stays the same. Only for EUAS1 (Eurasia mainland),
we observed that the two largest patches are always the same, implying
that these two large patches are produced by a geographical accident but
they have the same dynamics. The regions with \(RS_{max}\) less than
25\% SAST2 (Cuba) and EUAS3 (Great Britain) have an always-changing
largest patch reflecting their fragmented state. A transition is
observed in SEAS2 (Philippines), with the identity of the largest patch
first variable, and then constant after 2010.

The results of quantile regressions are almost identical for
\(\Delta RS_{max}\) and \(\Delta S_{max}\) (table S5). Among the biggest
regions, Africa (AF1) has a similar pattern across thresholds but only
the 30\% threshold is significant; the upper and lower quantiles have
significant negative slopes, but the lower quantile slope is lower,
implying that negative fluctuations and variance are increasing (Figure
4). Eurasia mainland (EUAS1) has significant slopes at 20\%, 30\% and
40\% thresholds but the patterns are different at 20\% variance is
decreasing, at 30\% and 40\% only is increasing. Thus the variation of
the most dense portion of the largest patch is increasing within a
limited range. North America mainland (NA1) exhibits the same pattern at
20\%, 25\% and 30\% thresholds: a significant lower quantile with
positive slope, implying decreasing variance. South America tropical and
subtropical (SAST1) have significant lower quantile with a negative
slope at 25\% and 30\% thresholds indicating an increase in variance.
SEAS1 has an upper quantile with a significant positive slope for 25\%
threshold, indicating an increasing variance. The other regions, with
forest area smaller than \(10^{7} \si{km^2}\) are shown in figure S11
and table S5. For Philippines (SEAS2), the slopes of lower quantiles are
positive for thresholds 20\% and 25\%, and the upper quantile slopes are
positive for thresholds 30\% and 40\%; thus variance is decreasing at
20\%-25\% and increasing at 30\%-40\%.

The conditions that indicate that a region is near a critical
fragmentation threshold are that patch size distributions follow a power
law; variance of \(\Delta RS_{max}\) is increasing in time, and skewness
is negative. All these conditions must happen at the same time at least
for one threshold. When the threshold is higher more dense regions of
the forest are at risk. This happens for Africa mainland (AF1), Eurasia
mainland(EUAS1), Japan (EUAS2), Australia mainland (OC1),
Malaysia/Kalimantan (OC3), Sumatra (OC4), South America tropical \&
subtropical (SAST1), Cuba (SAST2), Southeast Asia, Mainland (SEAS1).

\subsection{Discussion}\label{discussion}

We found that the forest patch distribution of all regions of the world,
spanning tropical rainforests, boreal and temperate forests, followed
power laws through seven orders of magnitude. Power laws have previously
been found for several kinds of vegetation, but never at global scales
as in this study. Moreover, the range of the estimated power law
exponents is relatively narrow (1.90 - 2.01), even though we used a
range of different thresholds levels. This suggests the existence of one
unifying mechanism that acts at continental scales, affecting forest
spatial structure and dynamics.

A possible mechanism for the emergence of power laws in forests is
isotropic percolation\textsuperscript{20}: forest sites disappear at
random positions when the density of forest is near the critical point,
and thus the power law structures arise. This requires the tuning of an
external environmental condition to carry the system to this point. We
did not expect forest growth to be a random process at local scales, but
it is possible that combinations of factors cancel out to produce
seemingly random forest dynamics at large scales. This has been
suggested as a mechanism for the observed power laws of global tropical
forest in year 2000\textsuperscript{21}. In this case, we should have
observed power laws in a limited set of situations that coincide with a
critical point, but instead, we observed pervasive power law
distributions. Thus isotropic percolation does not seem to be the
mechanism that produces the observed distributions.

Another possible mechanism is facilitation\textsuperscript{80,81}: a
patch surrounded by forest will have a smaller probability of being
deforested or degraded than an isolated patch. The model of Scanlon et
al.\textsuperscript{82} showed an \(\alpha=1.34\) which is far from our
results (1.90 - 2.01 range). Another model but with three states
(tree/non-tree/degraded), including local facilitation and grazing, has
also been used to obtain power laws patch distributions without external
tuning and exhibited deviations from power laws at high grazing
pressures\textsuperscript{83}. The values of the power law exponent
\(\alpha\) for this model depend on the intensity of facilitation: if it
is more intense the exponent is higher, but the maximal values they
obtained are still lower than the ones we observed. Thus an exploration
of the parameters of this model is needed to find if this is a plausible
mechanism.

A combination of spatial and temporal indicators is more reliable for
detecting critical transitions\textsuperscript{84}. We combined five
criteria to evaluate the closeness of the system to a fragmentation
threshold. Two were spatial: the forest patch size distribution, and the
proportion of the largest patch relative to total forest area
\(RS_{max}\). The other three were the distribution of temporal
fluctuations in the largest patch size, the trend in the variance, and
the skewness of the fluctuations. One of them, the distribution of
temporal fluctuations or \(\Delta RS_{max}\), cannot be applied with our
temporal resolution due to the difficulties of fitting and comparing
heavy-tailed distributions. The combination of the remaining four gives
us an increased degree of confidence about the system being close to a
critical transition.

Although our results suggest the existence of a unifying mechanism,
different factors could be acting in different regions and perhaps
different models are needed. As a consequence, there might exist various
critical fragmentation thresholds. As we did not elucidate the mechanism
and the factors that might be most important for each region, we cannot
determine the theoretical critical point, and this is why we tried to
find signals of critical transitions without knowing the exact value of
the critical fragmentation threshold.

South America tropical and subtropical (SAST1), Southeast Asia mainland
(SEAS1) and Africa mainland (AF1) met all criteria at least for one
threshold; these regions generally experience the biggest rates of
deforestation with a significant increase in loss of
forest\textsuperscript{54}. The most critical regions are Southeast Asia
and Africa, because the proportion of the largest patch relative to
total forest area \(RS_{max}\) was around 30\% thus they are in a
fragmented state. Due to our criteria for defining regions, we could not
detect the effect of conservation policies applied at a country level,
e.g.~the Natural Forest Conservation Program in China, which has
produced a 1.6\% increase in forest cover and net primary productivity
over the last 20 years\textsuperscript{85}. Tropical South America with
its high \(RS_{max}\) is also endangered but probably in an unfragmented
state. Indonesia and Malaysia (OC3) have both high deforestation
rates\textsuperscript{54}; Sumatra (OC4) is the biggest island of
Indonesia and where most deforestation occurs. Both regions show a high
\(RS_{max}\) greater than 60\%, and thus the forest is in an
unfragmented state, but they met all other criteria, meaning that they
are approaching a transition if current deforestation rates continue.

The Eurasian mainland region (EUAS1) is an extensive area with mainly
temperate and boreal forest and a combination of forest loss due to
fire\textsuperscript{86} and forestry. Russia, the biggest country, has
experienced the largest rate of forest loss of all countries, but in the
zone of coniferous forest, the largest gain is observed due to
agricultural abandonment\textsuperscript{87}. The loss is maximum at the
most dense areas of forest\textsuperscript{54}, and this coincides with
our analysis that detected an increased risk at denser forest. This
region also has a relatively low \(RS_{max}\) which means that it is
probably near a fragmented state. A possible explanation of this is that
in Russia after the collapse of the Soviet Union harvest was lower due
to agricultural abandonment, but illegal overharvesting of high valued
stands has increased in recent decades\textsuperscript{88}. A region
that is similar in forest composition to EAUS1 is North America (NA1);
the two main countries involved, United States and Canada, have forest
dynamics mainly influenced by fire and forestry, with both regions
extensively managed for industrial wood production. North America has a
higher \(RS_{max}\) than Eurasia and a positive skewness that excludes
it from being near a critical transition.

The analysis of \(RS_{max}\) reveals that the island of Philippines
(SEAS2) seems to be an example of a critical transition from an
unconnected to a connected state, i.e.~from a state with high
fluctuations and low \(RS_{max}\) to a state with low fluctuations and
high \(RS_{max}\). If we observe this pattern backwards in time, the
decrease in variance become an increase, and negative skewness is
constant, and thus the region exhibits the criteria of a critical
transition (Table 1, Figure S12). The actual pattern of transition to an
unfragmented state could be the result of an active intervention by the
government promoting conservation and rehabilitation of
forests\textsuperscript{89}. This confirms that the early warning
indicators proposed here work in the correct direction. An important
caveat is that the MODIS dataset does not detect if native forest is
replaced by agroindustrial tree plantations like oil palms, which are
among the main drivers of deforestation in this
area\textsuperscript{90}, for example in Indonesia and
Malaysia\textsuperscript{91}(Regions OC2,OC3, OC4, OC5, OC7). This
overestimates \(RS_{max}\) and in consequence, we underestimate the
fragmentation risks of these areas. To improve the estimation of forest
patches the Hansen's Landsat derived dataset\textsuperscript{54} should
be produced on a yearly basis. In addition, it would be important from a
conservation point of view to develop specific algorithms to detect
particular forest plantation types for each region---for example,
combining high-resolution images (e.g.~QuickBird 0.5m) with
change-detection of Landsat images\textsuperscript{91,92} to locate palm
oil plantations.

Deforestation and fragmentation are closely related. At low levels of
habitat reduction species population will decline proportionally; this
can happen even when the habitat fragments retain connectivity. As
habitat reduction continues, the critical threshold is approached and
connectivity will have large fluctuations\textsuperscript{93}. This
could trigger several negative synergistic effects: population
fluctuations and the possibility of extinctions will rise, increasing
patch isolation and decreasing connectivity\textsuperscript{93}. This
positive feedback mechanism will be enhanced when the fragmentation
threshold is reached, resulting in the loss of most habitat specialist
species at a landscape scale\textsuperscript{30}. If a forest is already
in a fragmented state, a second critical transition from forest to
non-forest could happen: the desertification
transition\textsuperscript{59}. Considering the actual trends of habitat
loss, and studying the dynamics of non-forest patches---instead of the
forest patches---the risk of this kind of transition could be estimated.
The simple models proposed previously could also be used to estimate if
these thresholds are likely to be continuous and reversible or
discontinuous and often irreversible\textsuperscript{94}, and the degree
of protection (e.g.~using the set-asides strategy 95) that would be
necessary to stop this trend.

Therefore, even if critical thresholds are reached only in some forest
regions at a continental scale, a cascading effect with global
consequences could still be produced\textsuperscript{96}. The risk of
such an event will be higher if the dynamics of separate continental
regions are coupled\textsuperscript{27}. At least three of the regions
defined here are considered tipping elements of the earth climate system
that could be triggered during this century\textsuperscript{97}. These
were defined as policy-relevant tipping elements so that political
decisions could determine whether the critical value is reached or not.
Thus the criteria proposed here could be used as a more sensitive system
to evaluate the closeness of a tipping point at a continental scale.
Further improvements will produce quantitative predictions about the
temporal horizon where these critical transitions could produce
significant changes in the studied systems.

\subsection{Figures captions}\label{figures-captions}

Figure 1: Forest patch distributions for continental regions for the
years 2000 and 2014. The images are the 200 biggest patches, shown at a
coarse pixel scale of 2.5 km. The regions are: a) \& b) southeast Asia;
c) \& d) South America subtropical and tropical and e) \& f) Africa
mainland, for the years 2000 and 2014 respectively. The color palette
was chosen to discriminate different patches and does not represent
patch size. The imaged was composed with GIMP 2.8 software and the base
maps retrieved from Google maps (Imagery(C)2018 NASA, TerraMetrics

Figure 2: Power law exponents (\(\alpha\)) of forest patch distributions
for regions with total forest area \(> 10^{7}\) \si{km^2}. Dashed
horizontal lines are the means by region, with 95\% confidence interval
error bars estimated by bootstrap resampling. The regions are AF1:
Africa mainland, EUAS1: Eurasia mainland, NA1: North America mainland,
SAST1: South America subtropical and tropical, SEAS1: Southeast Asia
mainland.

Figure 3: Largest patch proportion relative to total forest area
\(RS_{max}\), for regions with total forest area \(>10^{7}\) \si{km^2}.
We show here the \(RS_{max}\) calculated using a threshold of 40\% of
forest in each pixel to determine patches. Dashed lines are averages
across time. The regions are AF1: Africa mainland, EUAS1: Eurasia
mainland, NA1: North America mainland, SAST1: South America tropical and
subtropical, SEAS1: Southeast Asia mainland.

Figure 4: Largest patch fluctuations for regions with total forest area
\(>10^{7} \si{km^2}\) across years. The patch sizes are relative to the
total forest area of the same year. Dashed lines are 90\% and 10\%
quantile regressions, to show if fluctuations were increasing; purple
(dark) panels have significant slopes. The regions are AF1: Africa
mainland, EUAS1: Eurasia mainland, NA1: North America mainland, SAST1:
South America tropical and subtropical, SEAS1: Southeast Asia mainland.

\newpage

\subsection{Table 1}\label{table-1}

\scriptsize

\begin{longtable}[]{@{}clrccr@{}}
\caption{Regions and indicators of closeness to a critical fragmentation
point. Where: \(RS_{max}\) is the largest patch divided by the total
forest area; Threshold is the value used to calculate patches from the
MODIS VCF pixels; \(\Delta RS_{max}\) are the fluctuations of
\(RS_{max}\) around the mean and the increase or decrease in the
variance was estimated using quantile regressions; skewness was
calculated for \(RS_{max}\). NS means the results were non-significant.
The conditions that determine the closeness to a fragmentation point
are: increasing variance of \(\Delta RS_{max}\) and negative skewness.
\(RS_{max}\) indicates if the forest is unfragmented (\textgreater{}0.6)
or fragmented (\textless{}0.3).}\tabularnewline
\toprule
\begin{minipage}[b]{0.08\columnwidth}\centering\strut
Region\strut
\end{minipage} & \begin{minipage}[b]{0.29\columnwidth}\raggedright\strut
Description\strut
\end{minipage} & \begin{minipage}[b]{0.11\columnwidth}\raggedleft\strut
\(RS_{max}\)\strut
\end{minipage} & \begin{minipage}[b]{0.10\columnwidth}\centering\strut
Threshold\strut
\end{minipage} & \begin{minipage}[b]{0.16\columnwidth}\centering\strut
Variance of \(\Delta RS_{max}\)\strut
\end{minipage} & \begin{minipage}[b]{0.09\columnwidth}\raggedleft\strut
Skewness\strut
\end{minipage}\tabularnewline
\midrule
\endfirsthead
\toprule
\begin{minipage}[b]{0.08\columnwidth}\centering\strut
Region\strut
\end{minipage} & \begin{minipage}[b]{0.29\columnwidth}\raggedright\strut
Description\strut
\end{minipage} & \begin{minipage}[b]{0.11\columnwidth}\raggedleft\strut
\(RS_{max}\)\strut
\end{minipage} & \begin{minipage}[b]{0.10\columnwidth}\centering\strut
Threshold\strut
\end{minipage} & \begin{minipage}[b]{0.16\columnwidth}\centering\strut
Variance of \(\Delta RS_{max}\)\strut
\end{minipage} & \begin{minipage}[b]{0.09\columnwidth}\raggedleft\strut
Skewness\strut
\end{minipage}\tabularnewline
\midrule
\endhead
\begin{minipage}[t]{0.08\columnwidth}\centering\strut
AF1\strut
\end{minipage} & \begin{minipage}[t]{0.29\columnwidth}\raggedright\strut
Africa mainland\strut
\end{minipage} & \begin{minipage}[t]{0.11\columnwidth}\raggedleft\strut
0.33\strut
\end{minipage} & \begin{minipage}[t]{0.10\columnwidth}\centering\strut
30\strut
\end{minipage} & \begin{minipage}[t]{0.16\columnwidth}\centering\strut
Increase\strut
\end{minipage} & \begin{minipage}[t]{0.09\columnwidth}\raggedleft\strut
-1.4653\strut
\end{minipage}\tabularnewline
\begin{minipage}[t]{0.08\columnwidth}\centering\strut
AF2\strut
\end{minipage} & \begin{minipage}[t]{0.29\columnwidth}\raggedright\strut
Madagascar\strut
\end{minipage} & \begin{minipage}[t]{0.11\columnwidth}\raggedleft\strut
0.48\strut
\end{minipage} & \begin{minipage}[t]{0.10\columnwidth}\centering\strut
20\strut
\end{minipage} & \begin{minipage}[t]{0.16\columnwidth}\centering\strut
Increase\strut
\end{minipage} & \begin{minipage}[t]{0.09\columnwidth}\raggedleft\strut
-0.4461\strut
\end{minipage}\tabularnewline
\begin{minipage}[t]{0.08\columnwidth}\centering\strut
EUAS1\strut
\end{minipage} & \begin{minipage}[t]{0.29\columnwidth}\raggedright\strut
Eurasia, mainland\strut
\end{minipage} & \begin{minipage}[t]{0.11\columnwidth}\raggedleft\strut
0.22\strut
\end{minipage} & \begin{minipage}[t]{0.10\columnwidth}\centering\strut
20\strut
\end{minipage} & \begin{minipage}[t]{0.16\columnwidth}\centering\strut
Decrease\strut
\end{minipage} & \begin{minipage}[t]{0.09\columnwidth}\raggedleft\strut
-0.5015\strut
\end{minipage}\tabularnewline
\begin{minipage}[t]{0.08\columnwidth}\centering\strut
EUAS1\strut
\end{minipage} & \begin{minipage}[t]{0.29\columnwidth}\raggedright\strut
\strut
\end{minipage} & \begin{minipage}[t]{0.11\columnwidth}\raggedleft\strut
\strut
\end{minipage} & \begin{minipage}[t]{0.10\columnwidth}\centering\strut
30\strut
\end{minipage} & \begin{minipage}[t]{0.16\columnwidth}\centering\strut
Increase\strut
\end{minipage} & \begin{minipage}[t]{0.09\columnwidth}\raggedleft\strut
0.3113\strut
\end{minipage}\tabularnewline
\begin{minipage}[t]{0.08\columnwidth}\centering\strut
EUAS1\strut
\end{minipage} & \begin{minipage}[t]{0.29\columnwidth}\raggedright\strut
\strut
\end{minipage} & \begin{minipage}[t]{0.11\columnwidth}\raggedleft\strut
\strut
\end{minipage} & \begin{minipage}[t]{0.10\columnwidth}\centering\strut
40\strut
\end{minipage} & \begin{minipage}[t]{0.16\columnwidth}\centering\strut
Increase\strut
\end{minipage} & \begin{minipage}[t]{0.09\columnwidth}\raggedleft\strut
-1.316\strut
\end{minipage}\tabularnewline
\begin{minipage}[t]{0.08\columnwidth}\centering\strut
EUAS2\strut
\end{minipage} & \begin{minipage}[t]{0.29\columnwidth}\raggedright\strut
Japan\strut
\end{minipage} & \begin{minipage}[t]{0.11\columnwidth}\raggedleft\strut
0.94\strut
\end{minipage} & \begin{minipage}[t]{0.10\columnwidth}\centering\strut
35\strut
\end{minipage} & \begin{minipage}[t]{0.16\columnwidth}\centering\strut
Increase\strut
\end{minipage} & \begin{minipage}[t]{0.09\columnwidth}\raggedleft\strut
-0.3913\strut
\end{minipage}\tabularnewline
\begin{minipage}[t]{0.08\columnwidth}\centering\strut
EUAS2\strut
\end{minipage} & \begin{minipage}[t]{0.29\columnwidth}\raggedright\strut
\strut
\end{minipage} & \begin{minipage}[t]{0.11\columnwidth}\raggedleft\strut
\strut
\end{minipage} & \begin{minipage}[t]{0.10\columnwidth}\centering\strut
40\strut
\end{minipage} & \begin{minipage}[t]{0.16\columnwidth}\centering\strut
Increase\strut
\end{minipage} & \begin{minipage}[t]{0.09\columnwidth}\raggedleft\strut
-0.5030\strut
\end{minipage}\tabularnewline
\begin{minipage}[t]{0.08\columnwidth}\centering\strut
EUAS3\strut
\end{minipage} & \begin{minipage}[t]{0.29\columnwidth}\raggedright\strut
Great Britain\strut
\end{minipage} & \begin{minipage}[t]{0.11\columnwidth}\raggedleft\strut
0.03\strut
\end{minipage} & \begin{minipage}[t]{0.10\columnwidth}\centering\strut
40\strut
\end{minipage} & \begin{minipage}[t]{0.16\columnwidth}\centering\strut
NS\strut
\end{minipage} & \begin{minipage}[t]{0.09\columnwidth}\raggedleft\strut
0.1860\strut
\end{minipage}\tabularnewline
\begin{minipage}[t]{0.08\columnwidth}\centering\strut
NA1\strut
\end{minipage} & \begin{minipage}[t]{0.29\columnwidth}\raggedright\strut
North America, mainland\strut
\end{minipage} & \begin{minipage}[t]{0.11\columnwidth}\raggedleft\strut
0.31\strut
\end{minipage} & \begin{minipage}[t]{0.10\columnwidth}\centering\strut
20\strut
\end{minipage} & \begin{minipage}[t]{0.16\columnwidth}\centering\strut
Decrease\strut
\end{minipage} & \begin{minipage}[t]{0.09\columnwidth}\raggedleft\strut
-2.2895\strut
\end{minipage}\tabularnewline
\begin{minipage}[t]{0.08\columnwidth}\centering\strut
NA1\strut
\end{minipage} & \begin{minipage}[t]{0.29\columnwidth}\raggedright\strut
\strut
\end{minipage} & \begin{minipage}[t]{0.11\columnwidth}\raggedleft\strut
\strut
\end{minipage} & \begin{minipage}[t]{0.10\columnwidth}\centering\strut
25\strut
\end{minipage} & \begin{minipage}[t]{0.16\columnwidth}\centering\strut
Decrease\strut
\end{minipage} & \begin{minipage}[t]{0.09\columnwidth}\raggedleft\strut
-2.4465\strut
\end{minipage}\tabularnewline
\begin{minipage}[t]{0.08\columnwidth}\centering\strut
NA1\strut
\end{minipage} & \begin{minipage}[t]{0.29\columnwidth}\raggedright\strut
\strut
\end{minipage} & \begin{minipage}[t]{0.11\columnwidth}\raggedleft\strut
\strut
\end{minipage} & \begin{minipage}[t]{0.10\columnwidth}\centering\strut
30\strut
\end{minipage} & \begin{minipage}[t]{0.16\columnwidth}\centering\strut
Decrease\strut
\end{minipage} & \begin{minipage}[t]{0.09\columnwidth}\raggedleft\strut
-1.6340\strut
\end{minipage}\tabularnewline
\begin{minipage}[t]{0.08\columnwidth}\centering\strut
NA5\strut
\end{minipage} & \begin{minipage}[t]{0.29\columnwidth}\raggedright\strut
Newfoundland\strut
\end{minipage} & \begin{minipage}[t]{0.11\columnwidth}\raggedleft\strut
0.54\strut
\end{minipage} & \begin{minipage}[t]{0.10\columnwidth}\centering\strut
40\strut
\end{minipage} & \begin{minipage}[t]{0.16\columnwidth}\centering\strut
NS\strut
\end{minipage} & \begin{minipage}[t]{0.09\columnwidth}\raggedleft\strut
-0.1053\strut
\end{minipage}\tabularnewline
\begin{minipage}[t]{0.08\columnwidth}\centering\strut
OC1\strut
\end{minipage} & \begin{minipage}[t]{0.29\columnwidth}\raggedright\strut
Australia, Mainland\strut
\end{minipage} & \begin{minipage}[t]{0.11\columnwidth}\raggedleft\strut
0.36\strut
\end{minipage} & \begin{minipage}[t]{0.10\columnwidth}\centering\strut
30\strut
\end{minipage} & \begin{minipage}[t]{0.16\columnwidth}\centering\strut
Increase\strut
\end{minipage} & \begin{minipage}[t]{0.09\columnwidth}\raggedleft\strut
0.0920\strut
\end{minipage}\tabularnewline
\begin{minipage}[t]{0.08\columnwidth}\centering\strut
OC1\strut
\end{minipage} & \begin{minipage}[t]{0.29\columnwidth}\raggedright\strut
\strut
\end{minipage} & \begin{minipage}[t]{0.11\columnwidth}\raggedleft\strut
\strut
\end{minipage} & \begin{minipage}[t]{0.10\columnwidth}\centering\strut
35\strut
\end{minipage} & \begin{minipage}[t]{0.16\columnwidth}\centering\strut
Increase\strut
\end{minipage} & \begin{minipage}[t]{0.09\columnwidth}\raggedleft\strut
-0.8033\strut
\end{minipage}\tabularnewline
\begin{minipage}[t]{0.08\columnwidth}\centering\strut
OC2\strut
\end{minipage} & \begin{minipage}[t]{0.29\columnwidth}\raggedright\strut
New Guinea\strut
\end{minipage} & \begin{minipage}[t]{0.11\columnwidth}\raggedleft\strut
0.96\strut
\end{minipage} & \begin{minipage}[t]{0.10\columnwidth}\centering\strut
25\strut
\end{minipage} & \begin{minipage}[t]{0.16\columnwidth}\centering\strut
Decrease\strut
\end{minipage} & \begin{minipage}[t]{0.09\columnwidth}\raggedleft\strut
-0.1003\strut
\end{minipage}\tabularnewline
\begin{minipage}[t]{0.08\columnwidth}\centering\strut
OC2\strut
\end{minipage} & \begin{minipage}[t]{0.29\columnwidth}\raggedright\strut
\strut
\end{minipage} & \begin{minipage}[t]{0.11\columnwidth}\raggedleft\strut
\strut
\end{minipage} & \begin{minipage}[t]{0.10\columnwidth}\centering\strut
30\strut
\end{minipage} & \begin{minipage}[t]{0.16\columnwidth}\centering\strut
Decrease\strut
\end{minipage} & \begin{minipage}[t]{0.09\columnwidth}\raggedleft\strut
0.1214\strut
\end{minipage}\tabularnewline
\begin{minipage}[t]{0.08\columnwidth}\centering\strut
OC2\strut
\end{minipage} & \begin{minipage}[t]{0.29\columnwidth}\raggedright\strut
\strut
\end{minipage} & \begin{minipage}[t]{0.11\columnwidth}\raggedleft\strut
\strut
\end{minipage} & \begin{minipage}[t]{0.10\columnwidth}\centering\strut
35\strut
\end{minipage} & \begin{minipage}[t]{0.16\columnwidth}\centering\strut
Decrease\strut
\end{minipage} & \begin{minipage}[t]{0.09\columnwidth}\raggedleft\strut
-0.0124\strut
\end{minipage}\tabularnewline
\begin{minipage}[t]{0.08\columnwidth}\centering\strut
OC3\strut
\end{minipage} & \begin{minipage}[t]{0.29\columnwidth}\raggedright\strut
Malaysia/Kalimantan\strut
\end{minipage} & \begin{minipage}[t]{0.11\columnwidth}\raggedleft\strut
0.92\strut
\end{minipage} & \begin{minipage}[t]{0.10\columnwidth}\centering\strut
35\strut
\end{minipage} & \begin{minipage}[t]{0.16\columnwidth}\centering\strut
Increase\strut
\end{minipage} & \begin{minipage}[t]{0.09\columnwidth}\raggedleft\strut
-1.0147\strut
\end{minipage}\tabularnewline
\begin{minipage}[t]{0.08\columnwidth}\centering\strut
OC3\strut
\end{minipage} & \begin{minipage}[t]{0.29\columnwidth}\raggedright\strut
\strut
\end{minipage} & \begin{minipage}[t]{0.11\columnwidth}\raggedleft\strut
\strut
\end{minipage} & \begin{minipage}[t]{0.10\columnwidth}\centering\strut
40\strut
\end{minipage} & \begin{minipage}[t]{0.16\columnwidth}\centering\strut
Increase\strut
\end{minipage} & \begin{minipage}[t]{0.09\columnwidth}\raggedleft\strut
-1.5649\strut
\end{minipage}\tabularnewline
\begin{minipage}[t]{0.08\columnwidth}\centering\strut
OC4\strut
\end{minipage} & \begin{minipage}[t]{0.29\columnwidth}\raggedright\strut
Sumatra\strut
\end{minipage} & \begin{minipage}[t]{0.11\columnwidth}\raggedleft\strut
0.84\strut
\end{minipage} & \begin{minipage}[t]{0.10\columnwidth}\centering\strut
20\strut
\end{minipage} & \begin{minipage}[t]{0.16\columnwidth}\centering\strut
Increase\strut
\end{minipage} & \begin{minipage}[t]{0.09\columnwidth}\raggedleft\strut
-1.3846\strut
\end{minipage}\tabularnewline
\begin{minipage}[t]{0.08\columnwidth}\centering\strut
OC4\strut
\end{minipage} & \begin{minipage}[t]{0.29\columnwidth}\raggedright\strut
\strut
\end{minipage} & \begin{minipage}[t]{0.11\columnwidth}\raggedleft\strut
\strut
\end{minipage} & \begin{minipage}[t]{0.10\columnwidth}\centering\strut
25\strut
\end{minipage} & \begin{minipage}[t]{0.16\columnwidth}\centering\strut
Increase\strut
\end{minipage} & \begin{minipage}[t]{0.09\columnwidth}\raggedleft\strut
-0.5887\strut
\end{minipage}\tabularnewline
\begin{minipage}[t]{0.08\columnwidth}\centering\strut
OC4\strut
\end{minipage} & \begin{minipage}[t]{0.29\columnwidth}\raggedright\strut
\strut
\end{minipage} & \begin{minipage}[t]{0.11\columnwidth}\raggedleft\strut
\strut
\end{minipage} & \begin{minipage}[t]{0.10\columnwidth}\centering\strut
30\strut
\end{minipage} & \begin{minipage}[t]{0.16\columnwidth}\centering\strut
Increase\strut
\end{minipage} & \begin{minipage}[t]{0.09\columnwidth}\raggedleft\strut
-1.4226\strut
\end{minipage}\tabularnewline
\begin{minipage}[t]{0.08\columnwidth}\centering\strut
OC5\strut
\end{minipage} & \begin{minipage}[t]{0.29\columnwidth}\raggedright\strut
Sulawesi\strut
\end{minipage} & \begin{minipage}[t]{0.11\columnwidth}\raggedleft\strut
0.82\strut
\end{minipage} & \begin{minipage}[t]{0.10\columnwidth}\centering\strut
40\strut
\end{minipage} & \begin{minipage}[t]{0.16\columnwidth}\centering\strut
NS\strut
\end{minipage} & \begin{minipage}[t]{0.09\columnwidth}\raggedleft\strut
0.0323\strut
\end{minipage}\tabularnewline
\begin{minipage}[t]{0.08\columnwidth}\centering\strut
OC6\strut
\end{minipage} & \begin{minipage}[t]{0.29\columnwidth}\raggedright\strut
New Zealand South Island\strut
\end{minipage} & \begin{minipage}[t]{0.11\columnwidth}\raggedleft\strut
0.75\strut
\end{minipage} & \begin{minipage}[t]{0.10\columnwidth}\centering\strut
40\strut
\end{minipage} & \begin{minipage}[t]{0.16\columnwidth}\centering\strut
Increase\strut
\end{minipage} & \begin{minipage}[t]{0.09\columnwidth}\raggedleft\strut
0.3024\strut
\end{minipage}\tabularnewline
\begin{minipage}[t]{0.08\columnwidth}\centering\strut
OC7\strut
\end{minipage} & \begin{minipage}[t]{0.29\columnwidth}\raggedright\strut
Java\strut
\end{minipage} & \begin{minipage}[t]{0.11\columnwidth}\raggedleft\strut
0.16\strut
\end{minipage} & \begin{minipage}[t]{0.10\columnwidth}\centering\strut
40\strut
\end{minipage} & \begin{minipage}[t]{0.16\columnwidth}\centering\strut
NS\strut
\end{minipage} & \begin{minipage}[t]{0.09\columnwidth}\raggedleft\strut
2.0105\strut
\end{minipage}\tabularnewline
\begin{minipage}[t]{0.08\columnwidth}\centering\strut
OC8\strut
\end{minipage} & \begin{minipage}[t]{0.29\columnwidth}\raggedright\strut
New Zealand North Island\strut
\end{minipage} & \begin{minipage}[t]{0.11\columnwidth}\raggedleft\strut
0.64\strut
\end{minipage} & \begin{minipage}[t]{0.10\columnwidth}\centering\strut
40\strut
\end{minipage} & \begin{minipage}[t]{0.16\columnwidth}\centering\strut
NS\strut
\end{minipage} & \begin{minipage}[t]{0.09\columnwidth}\raggedleft\strut
1.3175\strut
\end{minipage}\tabularnewline
\begin{minipage}[t]{0.08\columnwidth}\centering\strut
SAST1\strut
\end{minipage} & \begin{minipage}[t]{0.29\columnwidth}\raggedright\strut
South America, Tropical and Subtropical forest\strut
\end{minipage} & \begin{minipage}[t]{0.11\columnwidth}\raggedleft\strut
0.56\strut
\end{minipage} & \begin{minipage}[t]{0.10\columnwidth}\centering\strut
25\strut
\end{minipage} & \begin{minipage}[t]{0.16\columnwidth}\centering\strut
Increase\strut
\end{minipage} & \begin{minipage}[t]{0.09\columnwidth}\raggedleft\strut
1.0519\strut
\end{minipage}\tabularnewline
\begin{minipage}[t]{0.08\columnwidth}\centering\strut
SAST1\strut
\end{minipage} & \begin{minipage}[t]{0.29\columnwidth}\raggedright\strut
\strut
\end{minipage} & \begin{minipage}[t]{0.11\columnwidth}\raggedleft\strut
\strut
\end{minipage} & \begin{minipage}[t]{0.10\columnwidth}\centering\strut
30\strut
\end{minipage} & \begin{minipage}[t]{0.16\columnwidth}\centering\strut
Increase\strut
\end{minipage} & \begin{minipage}[t]{0.09\columnwidth}\raggedleft\strut
-2.7216\strut
\end{minipage}\tabularnewline
\begin{minipage}[t]{0.08\columnwidth}\centering\strut
SAST2\strut
\end{minipage} & \begin{minipage}[t]{0.29\columnwidth}\raggedright\strut
Cuba\strut
\end{minipage} & \begin{minipage}[t]{0.11\columnwidth}\raggedleft\strut
0.15\strut
\end{minipage} & \begin{minipage}[t]{0.10\columnwidth}\centering\strut
20\strut
\end{minipage} & \begin{minipage}[t]{0.16\columnwidth}\centering\strut
Increase\strut
\end{minipage} & \begin{minipage}[t]{0.09\columnwidth}\raggedleft\strut
0.5049\strut
\end{minipage}\tabularnewline
\begin{minipage}[t]{0.08\columnwidth}\centering\strut
SAST2\strut
\end{minipage} & \begin{minipage}[t]{0.29\columnwidth}\raggedright\strut
\strut
\end{minipage} & \begin{minipage}[t]{0.11\columnwidth}\raggedleft\strut
\strut
\end{minipage} & \begin{minipage}[t]{0.10\columnwidth}\centering\strut
25\strut
\end{minipage} & \begin{minipage}[t]{0.16\columnwidth}\centering\strut
Increase\strut
\end{minipage} & \begin{minipage}[t]{0.09\columnwidth}\raggedleft\strut
1.7263\strut
\end{minipage}\tabularnewline
\begin{minipage}[t]{0.08\columnwidth}\centering\strut
SAST2\strut
\end{minipage} & \begin{minipage}[t]{0.29\columnwidth}\raggedright\strut
\strut
\end{minipage} & \begin{minipage}[t]{0.11\columnwidth}\raggedleft\strut
\strut
\end{minipage} & \begin{minipage}[t]{0.10\columnwidth}\centering\strut
30\strut
\end{minipage} & \begin{minipage}[t]{0.16\columnwidth}\centering\strut
Increase\strut
\end{minipage} & \begin{minipage}[t]{0.09\columnwidth}\raggedleft\strut
0.1665\strut
\end{minipage}\tabularnewline
\begin{minipage}[t]{0.08\columnwidth}\centering\strut
SAST2\strut
\end{minipage} & \begin{minipage}[t]{0.29\columnwidth}\raggedright\strut
\strut
\end{minipage} & \begin{minipage}[t]{0.11\columnwidth}\raggedleft\strut
\strut
\end{minipage} & \begin{minipage}[t]{0.10\columnwidth}\centering\strut
40\strut
\end{minipage} & \begin{minipage}[t]{0.16\columnwidth}\centering\strut
Increase\strut
\end{minipage} & \begin{minipage}[t]{0.09\columnwidth}\raggedleft\strut
-0.5401\strut
\end{minipage}\tabularnewline
\begin{minipage}[t]{0.08\columnwidth}\centering\strut
SAT1\strut
\end{minipage} & \begin{minipage}[t]{0.29\columnwidth}\raggedright\strut
South America, Temperate forest\strut
\end{minipage} & \begin{minipage}[t]{0.11\columnwidth}\raggedleft\strut
0.54\strut
\end{minipage} & \begin{minipage}[t]{0.10\columnwidth}\centering\strut
25\strut
\end{minipage} & \begin{minipage}[t]{0.16\columnwidth}\centering\strut
Decrease\strut
\end{minipage} & \begin{minipage}[t]{0.09\columnwidth}\raggedleft\strut
0.1483\strut
\end{minipage}\tabularnewline
\begin{minipage}[t]{0.08\columnwidth}\centering\strut
SAT1\strut
\end{minipage} & \begin{minipage}[t]{0.29\columnwidth}\raggedright\strut
\strut
\end{minipage} & \begin{minipage}[t]{0.11\columnwidth}\raggedleft\strut
\strut
\end{minipage} & \begin{minipage}[t]{0.10\columnwidth}\centering\strut
30\strut
\end{minipage} & \begin{minipage}[t]{0.16\columnwidth}\centering\strut
Decrease\strut
\end{minipage} & \begin{minipage}[t]{0.09\columnwidth}\raggedleft\strut
-1.6059\strut
\end{minipage}\tabularnewline
\begin{minipage}[t]{0.08\columnwidth}\centering\strut
SAT1\strut
\end{minipage} & \begin{minipage}[t]{0.29\columnwidth}\raggedright\strut
\strut
\end{minipage} & \begin{minipage}[t]{0.11\columnwidth}\raggedleft\strut
\strut
\end{minipage} & \begin{minipage}[t]{0.10\columnwidth}\centering\strut
35\strut
\end{minipage} & \begin{minipage}[t]{0.16\columnwidth}\centering\strut
Decrease\strut
\end{minipage} & \begin{minipage}[t]{0.09\columnwidth}\raggedleft\strut
-1.3809\strut
\end{minipage}\tabularnewline
\begin{minipage}[t]{0.08\columnwidth}\centering\strut
SEAS1\strut
\end{minipage} & \begin{minipage}[t]{0.29\columnwidth}\raggedright\strut
Southeast Asia, Mainland\strut
\end{minipage} & \begin{minipage}[t]{0.11\columnwidth}\raggedleft\strut
0.28\strut
\end{minipage} & \begin{minipage}[t]{0.10\columnwidth}\centering\strut
25\strut
\end{minipage} & \begin{minipage}[t]{0.16\columnwidth}\centering\strut
Increase\strut
\end{minipage} & \begin{minipage}[t]{0.09\columnwidth}\raggedleft\strut
-1.3328\strut
\end{minipage}\tabularnewline
\begin{minipage}[t]{0.08\columnwidth}\centering\strut
SEAS2\strut
\end{minipage} & \begin{minipage}[t]{0.29\columnwidth}\raggedright\strut
Philippines\strut
\end{minipage} & \begin{minipage}[t]{0.11\columnwidth}\raggedleft\strut
0.33\strut
\end{minipage} & \begin{minipage}[t]{0.10\columnwidth}\centering\strut
20\strut
\end{minipage} & \begin{minipage}[t]{0.16\columnwidth}\centering\strut
Decrease\strut
\end{minipage} & \begin{minipage}[t]{0.09\columnwidth}\raggedleft\strut
-1.6373\strut
\end{minipage}\tabularnewline
\begin{minipage}[t]{0.08\columnwidth}\centering\strut
SEAS2\strut
\end{minipage} & \begin{minipage}[t]{0.29\columnwidth}\raggedright\strut
\strut
\end{minipage} & \begin{minipage}[t]{0.11\columnwidth}\raggedleft\strut
\strut
\end{minipage} & \begin{minipage}[t]{0.10\columnwidth}\centering\strut
25\strut
\end{minipage} & \begin{minipage}[t]{0.16\columnwidth}\centering\strut
Decrease\strut
\end{minipage} & \begin{minipage}[t]{0.09\columnwidth}\raggedleft\strut
-0.6648\strut
\end{minipage}\tabularnewline
\begin{minipage}[t]{0.08\columnwidth}\centering\strut
SEAS2\strut
\end{minipage} & \begin{minipage}[t]{0.29\columnwidth}\raggedright\strut
\strut
\end{minipage} & \begin{minipage}[t]{0.11\columnwidth}\raggedleft\strut
\strut
\end{minipage} & \begin{minipage}[t]{0.10\columnwidth}\centering\strut
30\strut
\end{minipage} & \begin{minipage}[t]{0.16\columnwidth}\centering\strut
Increase\strut
\end{minipage} & \begin{minipage}[t]{0.09\columnwidth}\raggedleft\strut
0.1517\strut
\end{minipage}\tabularnewline
\begin{minipage}[t]{0.08\columnwidth}\centering\strut
SEAS2\strut
\end{minipage} & \begin{minipage}[t]{0.29\columnwidth}\raggedright\strut
\strut
\end{minipage} & \begin{minipage}[t]{0.11\columnwidth}\raggedleft\strut
\strut
\end{minipage} & \begin{minipage}[t]{0.10\columnwidth}\centering\strut
40\strut
\end{minipage} & \begin{minipage}[t]{0.16\columnwidth}\centering\strut
Increase\strut
\end{minipage} & \begin{minipage}[t]{0.09\columnwidth}\raggedleft\strut
1.5996\strut
\end{minipage}\tabularnewline
\bottomrule
\end{longtable}

\normalsize

\subsection{Data Accessibility}\label{data-accessibility}

The MODIS VCF product is freely available from NASA at
\url{https://search.earthdata.nasa.gov/}. Csv text file with model fits
for patch size distribution, and model selection for all the regions;
Gif Animations of a forest model percolation; Gif animations of largest
patches; patch size files for all years and regions used here; and all
the R, Python and Matlab scripts are available at figshare
\url{http://dx.doi.org/10.6084/m9.figshare.4263905}.

\subsection{Acknowledgments}\label{acknowledgments}

LAS and SRD are grateful to the National University of General Sarmiento
for financial support. We want to thank to Jordi Bascompte, Nara
Guisoni, Fernando Momo, and two anonymous reviewers for their comments
and discussions that greatly improved the manuscript. This work was
partially supported by a grant from CONICET (PIO 144-20140100035-CO).

\subsection*{References}\label{references}
\addcontentsline{toc}{subsection}{References}

\hypertarget{refs}{}
\hypertarget{ref-Crowther2015a}{}
1. Crowther, T. W. \emph{et al.} Mapping tree density at a global scale.
\emph{Nature} \textbf{525,} 201--205 (2015).

\hypertarget{ref-Canfield2010}{}
2. Canfield, D. E., Glazer, A. N. \& Falkowski, P. G. The Evolution and
Future of Earth's Nitrogen Cycle. \emph{Science} \textbf{330,} 192--196
(2010).

\hypertarget{ref-Bonan2008}{}
3. Bonan, G. B. Forests and Climate Change: Forcings, Feedbacks, and the
Climate Benefits of Forests. \emph{Science} \textbf{320,} 1444--1449
(2008).

\hypertarget{ref-Foley2011}{}
4. Foley, J. A. \emph{et al.} Solutions for a cultivated planet.
\emph{Nature} \textbf{478,} 337--342 (2011).

\hypertarget{ref-Barnosky2012}{}
5. Barnosky, A. D. \emph{et al.} Approaching a state shift in Earth's
biosphere. \emph{Nature} \textbf{486,} 52--58 (2012).

\hypertarget{ref-Haddad2015}{}
6. Haddad, N. M. \emph{et al.} Habitat fragmentation and its lasting
impact on Earth's ecosystems. \emph{Science Advances} \textbf{1,} 1--9
(2015).

\hypertarget{ref-Mitchell2015}{}
7. Mitchell, M. G. E. \emph{et al.} Reframing landscape fragmentation's
effects on ecosystem services. \emph{Trends in Ecology \& Evolution}
\textbf{30,} 190--198 (2015).

\hypertarget{ref-Angelsen2010}{}
8. Angelsen, A. Policies for reduced deforestation and their impact on
agricultural production. \emph{Proceedings of the National Academy of
Sciences} \textbf{107,} 19639--19644 (2010).

\hypertarget{ref-Rudel2005}{}
9. Rudel, T. K. \emph{et al.} Forest transitions: towards a global
understanding of land use change. \emph{Global Environmental Change}
\textbf{15,} 23--31 (2005).

\hypertarget{ref-Gilman2010}{}
10. Gilman, S. E., Urban, M. C., Tewksbury, J., Gilchrist, G. W. \&
Holt, R. D. A framework for community interactions under climate change.
\emph{Trends in Ecology \& Evolution} \textbf{25,} 325--331 (2010).

\hypertarget{ref-Rietkerk2011}{}
11. Rietkerk, M. \emph{et al.} Local ecosystem feedbacks and critical
transitions in the climate. \emph{Ecological Complexity} \textbf{8,}
223--228 (2011).

\hypertarget{ref-Scheffer2009}{}
12. Scheffer, M. \emph{et al.} Early-warning signals for critical
transitions. \emph{Nature} \textbf{461,} 53--59 (2009).

\hypertarget{ref-Sole2011}{}
13. Solé, R. V. \emph{Phase Transitions}. 223 (Princeton University
Press, 2011).

\hypertarget{ref-Scheffer2001}{}
14. Scheffer, M. \emph{et al.} Catastrophic shifts in ecosystems.
\emph{Nature} \textbf{413,} 591--596 (2001).

\hypertarget{ref-Bestelmeyer2011}{}
15. Bestelmeyer, B. T. \emph{et al.} Analysis of abrupt transitions in
ecological systems. \emph{Ecosphere} \textbf{2,} 129 (2011).

\hypertarget{ref-Vasilakopoulos2015}{}
16. Vasilakopoulos, P. \& Marshall, C. T. Resilience and tipping points
of an exploited fish population over six decades. \emph{Global Change
Biology} \textbf{21,} 1834--1847 (2015).

\hypertarget{ref-VillaMartin2015}{}
17. Villa Martín, P., Bonachela, J. A., Levin, S. A. \& Muñoz, M. A.
Eluding catastrophic shifts. \emph{Proceedings of the National Academy
of Sciences} \textbf{112,} E1828--E1836 (2015).

\hypertarget{ref-Kitzberger2012}{}
18. Kitzberger, T., Aráoz, E., Gowda, J. H., Mermoz, M. \& Morales, J.
M. Decreases in Fire Spread Probability with Forest Age Promotes
Alternative Community States, Reduced Resilience to Climate Variability
and Large Fire Regime Shifts. \emph{Ecosystems} \textbf{15,} 97--112
(2012).

\hypertarget{ref-Sole2006}{}
19. Solé, R. V. \& Bascompte, J. \emph{Self-organization in complex
ecosystems}. 373 (Princeton University Press, 2006).

\hypertarget{ref-Pueyo2010}{}
20. Pueyo, S. \emph{et al.} Testing for criticality in ecosystem
dynamics: the case of Amazonian rainforest and savanna fire.
\emph{Ecology Letters} \textbf{13,} 793--802 (2010).

\hypertarget{ref-Taubert2018}{}
21. Taubert, F. \emph{et al.} Global patterns of tropical forest
fragmentation. \emph{Nature} (2018).

\hypertarget{ref-McKenzie2012}{}
22. McKenzie, D. \& Kennedy, M. C. Power laws reveal phase transitions
in landscape controls of fire regimes. \emph{Nat Commun} \textbf{3,} 726
(2012).

\hypertarget{ref-Naito2015}{}
23. Naito, A. T. \& Cairns, D. M. Patterns of shrub expansion in Alaskan
arctic river corridors suggest phase transition. \emph{Ecology and
Evolution} \textbf{5,} 87--101 (2015).

\hypertarget{ref-Stauffer1994}{}
24. Stauffer, D. \& Aharony, A. \emph{Introduction To Percolation
Theory}. 179 (Tayor \& Francis, 1994).

\hypertarget{ref-Ochoa-Quintero2015}{}
25. Ochoa-Quintero, J. M., Gardner, T. A., Rosa, I., de Barros Ferraz,
S. F. \& Sutherland, W. J. Thresholds of species loss in Amazonian
deforestation frontier landscapes. \emph{Conservation Biology}
\textbf{29,} 440--451 (2015).

\hypertarget{ref-Leibold2004a}{}
26. Leibold, M. A. \& Norberg, J. Biodiversity in metacommunities:
Plankton as complex adaptive systems? \emph{Limnology and Oceanography}
\textbf{49,} 1278--1289 (2004).

\hypertarget{ref-Lenton2013}{}
27. Lenton, T. M. \& Williams, H. T. P. On the origin of planetary-scale
tipping points. \emph{Trends in Ecology \& Evolution} \textbf{28,}
380--382 (2013).

\hypertarget{ref-Sole2005}{}
28. Solé, R. V., Bartumeus, F. \& Gamarra, J. G. P. Gap percolation in
rainforests. \emph{Oikos} \textbf{110,} 177--185 (2005).

\hypertarget{ref-He2003}{}
29. He, F. \& Hubbell, S. Percolation Theory for the Distribution and
Abundance of Species. \emph{Physical Review Letters} \textbf{91,} 198103
(2003).

\hypertarget{ref-Pardini2010}{}
30. Pardini, R., Bueno, A. de A., Gardner, T. A., Prado, P. I. \&
Metzger, J. P. Beyond the Fragmentation Threshold Hypothesis: Regime
Shifts in Biodiversity Across Fragmented Landscapes. \emph{PLoS ONE}
\textbf{5,} e13666 (2010).

\hypertarget{ref-Bascompte1996}{}
31. Bascompte, J. \& Solé, R. V. Habitat fragmentation and extinction
threholds in spatially explicit models. \emph{Journal of Animal Ecology}
\textbf{65,} 465--473 (1996).

\hypertarget{ref-Sole2004}{}
32. Solé, R. V., Alonso, D. \& Saldaña, J. Habitat fragmentation and
biodiversity collapse in neutral communities. \emph{Ecological
Complexity} \textbf{1,} 65--75 (2004).

\hypertarget{ref-Loehle1996b}{}
33. Loehle, C., Li, B.-L. \& Sundell, R. C. Forest spread and phase
transitions at forest-praire ecotones in Kansas, U.S.A. \emph{Landscape
Ecology} \textbf{11,} 225--235 (1996).

\hypertarget{ref-Carpenter2011}{}
34. Carpenter, S. R. \emph{et al.} Early Warnings of Regime Shifts: A
Whole-Ecosystem Experiment. \emph{Science} \textbf{332,} 1079--1082
(2011).

\hypertarget{ref-Dai2012}{}
35. Dai, L., Vorselen, D., Korolev, K. S. \& Gore, J. Generic Indicators
for Loss of Resilience Before a Tipping Point Leading to Population
Collapse. \emph{Science} \textbf{336,} 1175--1177 (2012).

\hypertarget{ref-Hastings2010b}{}
36. Hastings, A. \& Wysham, D. B. Regime shifts in ecological systems
can occur with no warning. \emph{Ecology Letters} \textbf{13,} 464--472
(2010).

\hypertarget{ref-Boettiger2012}{}
37. Boettiger, C. \& Hastings, A. Quantifying limits to detection of
early warning for critical transitions. \emph{Journal of The Royal
Society Interface} \textbf{9,} 2527--2539 (2012).

\hypertarget{ref-Hirota2011}{}
38. Hirota, M., Holmgren, M., Nes, E. H. V. \& Scheffer, M. Global
Resilience of Tropical Forest and Savanna to Critical Transitions.
\emph{Science} \textbf{334,} 232--235 (2011).

\hypertarget{ref-Staal2016}{}
39. Staal, A., Dekker, S. C., Xu, C. \& Nes, E. H. van. Bistability,
Spatial Interaction, and the Distribution of Tropical Forests and
Savannas. \emph{Ecosystems} \textbf{19,} 1080--1091 (2016).

\hypertarget{ref-Wuyts2017}{}
40. Wuyts, B., Champneys, A. R. \& House, J. I. Amazonian forest-savanna
bistability and human impact. \emph{Nature Communications} \textbf{8,}
15519 (2017).

\hypertarget{ref-Filotas2014}{}
41. Filotas, E. \emph{et al.} Viewing forests through the lens of
complex systems science. \emph{Ecosphere} \textbf{5,} 1--23 (2014).

\hypertarget{ref-Fung2016}{}
42. Fung, T., O'Dwyer, J. P., Rahman, K. A., Fletcher, C. D. \&
Chisholm, R. A. Reproducing static and dynamic biodiversity patterns in
tropical forests: the critical role of environmental variance.
\emph{Ecology} \textbf{97,} 1207--1217 (2016).

\hypertarget{ref-Seidler2006}{}
43. Seidler, T. G. \& Plotkin, J. B. Seed Dispersal and Spatial Pattern
in Tropical Trees. \emph{PLoS Biology} \textbf{4,} e344 (2006).

\hypertarget{ref-Zhang2005}{}
44. Zhang, J. Y., Wang, Y., Zhao, X., Xie, G. \& Zhang, T. Grassland
recovery by protection from grazing in a semi‐arid sandy region of
northern China. \emph{New Zealand Journal of Agricultural Research}
\textbf{48,} 277--284 (2005).

\hypertarget{ref-Bestelmeyer2013}{}
45. Bestelmeyer, B. T., Duniway, M. C., James, D. K., Burkett, L. M. \&
Havstad, K. M. A test of critical thresholds and their indicators in a
desertification-prone ecosystem: more resilience than we thought.
\emph{Ecology Letters} \textbf{16,} 339--345 (2013).

\hypertarget{ref-Allington2010}{}
46. Allington, G. R. H. \& Valone, T. J. Reversal of desertification:
The role of physical and chemical soil properties. \emph{Journal of Arid
Environments} \textbf{74,} 973--977 (2010).

\hypertarget{ref-DiMiceli2015}{}
47. DiMiceli, C. \emph{et al.} Annual Global Automated MODIS Vegetation
Continuous Fields (MOD44B) at 250 m Spatial Resolution for Data Years
Beginning Day 65, 2000 - 2014, Collection 051 Percent Tree Cover,
University of Maryland, College Park, MD, USA. (2015).

\hypertarget{ref-Sexton2015}{}
48. Sexton, J. O. \emph{et al.} Conservation policy and the measurement
of forests. \emph{Nature Climate Change} \textbf{6,} 192--196 (2015).

\hypertarget{ref-Belward1996}{}
49. Belward, A. S. \emph{The IGBP-DIS Global 1 Km Land Cover Data Set
`DISCover': Proposal and Implementation Plans : Report of the Land
Recover Working Group of IGBP-DIS}. 61 (IGBP-DIS Office, 1996).

\hypertarget{ref-Xu2016}{}
50. Xu, C. \emph{et al.} Remotely sensed canopy height reveals three
pantropical ecosystem states. \emph{Ecology} \textbf{97,} 2518--2521
(2016).

\hypertarget{ref-Sexton2013}{}
51. Sexton, J. O. \emph{et al.} Global, 30-m resolution continuous
fields of tree cover: Landsat-based rescaling of MODIS vegetation
continuous fields with lidar-based estimates of error.
\emph{International Journal of Digital Earth} \textbf{6,} 427--448
(2013).

\hypertarget{ref-Hansen2014}{}
52. Hansen, M. \emph{et al.} Response to Comment on `High-resolution
global maps of 21st-century forest cover change'. \emph{Science}
\textbf{344,} 981 (2014).

\hypertarget{ref-Chaplin-Kramer2015}{}
53. Chaplin-Kramer, R. \emph{et al.} Degradation in carbon stocks near
tropical forest edges. \emph{Nature Communications} \textbf{6,} 10158
(2015).

\hypertarget{ref-Hansen2013}{}
54. Hansen, M. C. \emph{et al.} High-Resolution Global Maps of
21st-Century Forest Cover Change. \emph{Science} \textbf{342,} 850--853
(2013).

\hypertarget{ref-Oborny2007}{}
55. Oborny, B., Szabó, G. \& Meszéna, G. Survival of species in patchy
landscapes: percolation in space and time. in \emph{Scaling
biodiversity} 409--440 (Cambridge University Press, 2007).

\hypertarget{ref-Gastner2009}{}
56. Gastner, M. T., Oborny, B., Zimmermann, D. K. \& Pruessner, G.
Transition from Connected to Fragmented Vegetation across an
Environmental Gradient: Scaling Laws in Ecotone Geometry. \emph{The
American Naturalist} \textbf{174,} E23--E39 (2009).

\hypertarget{ref-Odor2004}{}
57. Ódor, G. Universality classes in nonequilibrium lattice systems.
\emph{Reviews of Modern Physics} \textbf{76,} 663--724 (2004).

\hypertarget{ref-Hinrichsen2000}{}
58. Hinrichsen, H. Non-equilibrium critical phenomena and phase
transitions into absorbing states. \emph{Advances in Physics}
\textbf{49,} 815--958 (2000).

\hypertarget{ref-Corrado2014}{}
59. Corrado, R., Cherubini, A. M. \& Pennetta, C. Early warning signals
of desertification transitions in semiarid ecosystems. \emph{Physical
Review E - Statistical, Nonlinear, and Soft Matter Physics} \textbf{90,}
62705 (2014).

\hypertarget{ref-Harris1974}{}
60. Harris, T. E. Contact interactions on a lattice. \emph{The Annals of
Probability} \textbf{2,} 969--988 (1974).

\hypertarget{ref-Goldstein2004}{}
61. Goldstein, M. L., Morris, S. A. \& Yen, G. G. Problems with fitting
to the power-law distribution. \emph{The European Physical Journal B -
Condensed Matter and Complex Systems} \textbf{41,} 255--258 (2004).

\hypertarget{ref-Clauset2009}{}
62. Clauset, A., Shalizi, C. \& Newman, M. Power-Law Distributions in
Empirical Data. \emph{SIAM Review} \textbf{51,} 661--703 (2009).

\hypertarget{ref-Limpert2001}{}
63. Limpert, E., Stahel, W. A. \& Abbt, M. Log-normal Distributions
across the Sciences: Keys and CluesOn the charms of statistics, and how
mechanical models resembling gambling machines offer a link to a handy
way to characterize log-normal distributions, which can provide deeper
insight into var. \emph{BioScience} \textbf{51,} 341--352 (2001).

\hypertarget{ref-Burnham2002}{}
64. Burnham, K. \& Anderson, D. R. \emph{Model selection and multi-model
inference: A practical information-theoretic approach}. 512
(Springer-Verlag, 2002).

\hypertarget{ref-Vuong1989}{}
65. Vuong, Q. H. Likelihood Ratio Tests for Model Selection and
Non-Nested Hypotheses. \emph{Econometrica} \textbf{57,} 307--333 (1989).

\hypertarget{ref-Crawley2012}{}
66. Crawley, M. J. \emph{The R Book}. 1076 (Wiley, 2012).

\hypertarget{ref-Efron1994}{}
67. Efron, B. \& Tibshirani, R. J. \emph{An Introduction to the
Bootstrap}. 456 (Taylor \& Francis, 1994).

\hypertarget{ref-Gardner2007}{}
68. Gardner, R. H. \& Urban, D. L. Neutral models for testing landscape
hypotheses. \emph{Landscape Ecology} \textbf{22,} 15--29 (2007).

\hypertarget{ref-Bazant2000}{}
69. Bazant, M. Z. Largest cluster in subcritical percolation.
\emph{Physical Review E} \textbf{62,} 1660--1669 (2000).

\hypertarget{ref-Botet2004}{}
70. Botet, R. \& Ploszajczak, M. Correlations in Finite Systems and
Their Universal Scaling Properties. in \emph{Nonequilibrium physics at
short time scales: Formation of correlations} (ed. Morawetz, K.)
445--466 (Springer-Verlag, 2004).

\hypertarget{ref-Keitt1997}{}
71. Keitt, T. H., Urban, D. L. \& Milne, B. T. Detecting critical scales
in fragmented landscapes. \emph{Conservation Ecology} \textbf{1,} 4
(1997).

\hypertarget{ref-Saravia2018}{}
72. Saravia, L. A. \& Momo, F. R. Biodiversity collapse and early
warning indicators in a spatial phase transition between neutral and
niche communities. \emph{Oikos} \textbf{127,} 111--124 (2018).

\hypertarget{ref-Hartigan1985}{}
73. Hartigan, J. A. \& Hartigan, P. M. The Dip Test of Unimodality.
\emph{The Annals of Statistics} \textbf{13,} 70--84 (1985).

\hypertarget{ref-VanRooij2013}{}
74. Rooij, M. M. J. W. van, Nash, B., Rajaraman, S. \& Holden, J. G. A
Fractal Approach to Dynamic Inference and Distribution Analysis.
\emph{Frontiers in Physiology} \textbf{4,} (2013).

\hypertarget{ref-Benedetti-Cecchi2015}{}
75. Benedetti-Cecchi, L., Tamburello, L., Maggi, E. \& Bulleri, F.
Experimental Perturbations Modify the Performance of Early Warning
Indicators of Regime Shift. \emph{Current biology} \textbf{25,}
1867--1872 (2015).

\hypertarget{ref-RCoreTeam2015}{}
76. R Core Team. R: A Language and Environment for Statistical
Computing. (2015).

\hypertarget{ref-Alstott2014}{}
77. Alstott, J., Bullmore, E. \& Plenz, D. powerlaw: A Python Package
for Analysis of Heavy-Tailed Distributions. \emph{PLOS ONE} \textbf{9,}
e85777 (2014).

\hypertarget{ref-Koenker2016}{}
78. Koenker, R. quantreg: Quantile Regression. (2016).

\hypertarget{ref-Newman2005}{}
79. Newman, M. E. J. Power laws, Pareto distributions and Zipf's law.
\emph{Contemporary Physics} \textbf{46,} 323--351 (2005).

\hypertarget{ref-Manor2008a}{}
80. Manor, A. \& Shnerb, N. M. Origin of pareto-like spatial
distributions in ecosystems. \emph{Physical Review Letters}
\textbf{101,} 268104 (2008).

\hypertarget{ref-Irvine2016}{}
81. Irvine, M. A., Bull, J. C. \& Keeling, M. J. Aggregation dynamics
explain vegetation patch-size distributions. \emph{Theoretical
Population Biology} \textbf{108,} 70--74 (2016).

\hypertarget{ref-Scanlon2007}{}
82. Scanlon, T. M., Caylor, K. K., Levin, S. A. \& Rodriguez-iturbe, I.
Positive feedbacks promote power-law clustering of Kalahari vegetation.
\emph{Nature} \textbf{449,} 209--212 (2007).

\hypertarget{ref-Kefi2007b}{}
83. Kéfi, S. \emph{et al.} Spatial vegetation patterns and imminent
desertification in Mediterranean arid ecosystems. \emph{Nature}
\textbf{449,} 213--217 (2007).

\hypertarget{ref-Kefi2014}{}
84. Kéfi, S. \emph{et al.} Early Warning Signals of Ecological
Transitions: Methods for Spatial Patterns. \emph{PLoS ONE} \textbf{9,}
e92097 (2014).

\hypertarget{ref-Vina2016}{}
85. Viña, A., McConnell, W. J., Yang, H., Xu, Z. \& Liu, J. Effects of
conservation policy on China's forest recovery. \emph{Science Advances}
\textbf{2,} e1500965 (2016).

\hypertarget{ref-Potapov2008a}{}
86. Potapov, P., Hansen, M. C., Stehman, S. V., Loveland, T. R. \&
Pittman, K. Combining MODIS and Landsat imagery to estimate and map
boreal forest cover loss. \emph{Remote Sensing of Environment}
\textbf{112,} 3708--3719 (2008).

\hypertarget{ref-Prishchepov2013}{}
87. Prishchepov, A. V., Müller, D., Dubinin, M., Baumann, M. \&
Radeloff, V. C. Determinants of agricultural land abandonment in
post-Soviet European Russia. \emph{Land Use Policy} \textbf{30,}
873--884 (2013).

\hypertarget{ref-Gauthier2015}{}
88. Gauthier, S., Bernier, P., Kuuluvainen, T., Shvidenko, A. Z. \&
Schepaschenko, D. G. Boreal forest health and global change.
\emph{Science} \textbf{349,} 819 LP--822 (2015).

\hypertarget{ref-Lasco2008}{}
89. Lasco, R. D. \emph{et al.} Forest responses to changing rainfall in
the Philippines. in \emph{Climate change and vulnerability} (eds. Leary,
N., Conde, C. \& Kulkarni, J.) 49--66 (Earthscan, 2008).

\hypertarget{ref-Malhi2014}{}
90. Malhi, Y., Gardner, T. A., Goldsmith, G. R., Silman, M. R. \&
Zelazowski, P. Tropical Forests in the Anthropocene. \emph{Annual Review
of Environment and Resources} \textbf{39,} 125--159 (2014).

\hypertarget{ref-Chong2017}{}
91. Chong, K. L., Kanniah, K. D., Pohl, C. \& Tan, K. P. A review of
remote sensing applications for oil palm studies. \emph{Geo-spatial
Information Science} \textbf{20,} 184--200 (2017).

\hypertarget{ref-Buchanan2008}{}
92. Buchanan, G. M. \emph{et al.} Using remote sensing to inform
conservation status assessment: Estimates of recent deforestation rates
on New Britain and the impacts upon endemic birds. \emph{Biological
Conservation} \textbf{141,} 56--66 (2008).

\hypertarget{ref-Brook2013}{}
93. Brook, B. W., Ellis, E. C., Perring, M. P., Mackay, A. W. \&
Blomqvist, L. Does the terrestrial biosphere have planetary tipping
points? \emph{Trends in Ecology \& Evolution} (2013).
doi:\href{https://doi.org/10.1016/j.tree.2013.01.016}{10.1016/j.tree.2013.01.016}

\hypertarget{ref-Weissmann2016}{}
94. Weissmann, H. \& Shnerb, N. M. Predicting catastrophic shifts.
\emph{Journal of Theoretical Biology} \textbf{397,} 128--134 (2016).

\hypertarget{ref-Banks-Leite2014}{}
95. Banks-Leite, C. \emph{et al.} Using ecological thresholds to
evaluate the costs and benefits of set-asides in a biodiversity hotspot.
\emph{Science} \textbf{345,} 1041--1045 (2014).

\hypertarget{ref-Reyer2015}{}
96. Reyer, C. P. O., Rammig, A., Brouwers, N. \& Langerwisch, F. Forest
resilience, tipping points and global change processes. \emph{Journal of
Ecology} \textbf{103,} 1--4 (2015).

\hypertarget{ref-Lenton2008}{}
97. Lenton, T. M. \emph{et al.} Tipping elements in the Earth's climate
system. \emph{Proceedings of the National Academy of Sciences}
\textbf{105,} 1786--1793 (2008).

\end{document}
